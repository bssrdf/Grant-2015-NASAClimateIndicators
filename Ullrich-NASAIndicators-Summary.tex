The next century will see unprecedented changes to the climate system, which are in turn expected to drive changes to the characteristics of meteorological extremes. These changes will have direct impact on populations throughout the United States; as stated by third National Climate Assessment (NCA), ``changes in extreme weather and climate events, such as heat waves and droughts, are the primary way that most people experience climate change.'' The characteristics of extreme weather are key climate indicators, and assessing trends in these quantities will be important across multiple sectors, in multiple regions, and highly useful input to future NCAs. This project tackles the need for a better understanding of changing extremes, including tropical cyclones (TCs), extratropical cyclones (ETCs), atmospheric blocks, atmospheric rivers (ARs), temperature extremes and precipitation extremes. Examples of characteristics include, for TCs and ETCs, wind / precipitation intensity and track density; for blocks, intensity, duration and location; for atmospheric rivers, precipitation intensity and point of landfall.

The capability to address such a wide array of extremes hinges on the TempestExtremes software package, a new suite of flexible detection and characterization algorithms developed by the PI for processing large climate datasets. This package uses an algorithmic framework known as "MapReduce" to first detect candidate events at individual times using specified criteria. Stitching is then used to assess the evolution of related detections over time. The result is an objective calculation of the climate indicator that can be automated and parallelized for multiple datasets.

Deliverables:  This project will provide a catalogue of historical extremes and their characteristics, and an assessment of observed trends in these characteristics. This catalogue will be built from NASA climate datasets and other major climate products. This proposal also aims to address the question of attribution via the project "Assessing Human-Induced Climate Change", described below.  To support future assessment capabilities, this project further seeks to integrate the TempestExtremes package in the NASA Earth Exchange (NEX) to allow for high-throughput analysis of submitted datasets from other NEX users.

NASA Datasets:  NASA climate products will be key to assess past changes, including the Modern Era Retrospective-analysis for Research and Applications (MERRA) and high-resolution Daymet dataset (for temperature and precipitation extremes)

Other datasets:  This project will also examine other major climate products, including Era-Interim reanalysis, NCEP CFSR, CMIP5 hindcast simulations, North American Regional Reanalysis (NARR) and PRISM precipitation data (for precipitation extremes).

Assessing Human-Induced Climate Change:  As part of the Climate of the 20th Century (C20C) project, 100 climate simulations have been produced covering the period 1959-2011, half representing a climate consistent with known greenhouse gas emissions and sea-surface temperatures, and half representing conditions with anthropogenic forcing removed.  By contrasting extreme weather events within these two datasets, anthropogenic influences can be separated from natural variability. These ensemble runs significantly enlarge the sample of extreme events, which are inherently rare. Hence these data are particularly useful when performing a statistical analysis of the time series of observation-based historical data and improving statistical estimates of use to various application sectors.
