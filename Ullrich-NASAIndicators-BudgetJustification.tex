\documentclass[11pt]{article}

\usepackage{amsmath}
\usepackage{graphicx}
\usepackage{multicol}
\usepackage{natbib}
\usepackage{wrapfig}
\usepackage{hyperref}
\usepackage{tabularx}
\usepackage{setspace}
\usepackage{comment}

\oddsidemargin 0cm
\evensidemargin 0cm

\usepackage[margin=1in]{geometry}

\usepackage[compact]{titlesec}  
\titlespacing{\section}{0pt}{6pt}{6pt}

\parindent 0cm
\parskip 0.5cm

\usepackage{fancyhdr}
\pagestyle{plain}
%\fancyhf{}
%\fancyhead[L]{AOSS Reference Sheet}
%\fancyhead[CH]{test}
\fancyfoot[C]{Page \thepage}

\newcommand{\vb}{\mathbf}
\newcommand{\diff}[2]{\frac{d #1}{d #2}}
\newcommand{\diffsq}[2]{\frac{d^2 #1}{{d #2}^2}}
\newcommand{\pdiff}[2]{\frac{\partial #1}{\partial #2}}
\newcommand{\pdiffsq}[2]{\frac{\partial^2 #1}{{\partial #2}^2}}
\newcommand{\topic}{\textbf}
\newcommand{\arcsinh}{\mathrm{arcsinh}}
\newcommand{\arccosh}{\mathrm{arccosh}}
\newcommand{\arctanh}{\mathrm{arctanh}}


\begin{document}

{\large \textbf{TempestExtremes: Indicators of change in the characteristics of extreme weather}}

\textbf{PI:} Dr. Paul Ullrich, University of California, Davis, CA

\textbf{Co-PI:} Dr. Richard Grotjahn, University of California, Davis, CA

\textbf{Co-I:} Dr. Colin Zarzycki, National Center for Atmospheric Research, Boulder, CO

\vspace{0.3cm}

{\Large \textbf{Budget Justification}}

\section{Inflation}
\vspace{-0.3cm}

The inflation rate is assumed to be 3\% per year on salaries.

\section{Salaries}
\vspace{-0.3cm}

\textit{Principal Investigators} \\
PI Paul Ullrich will be reimbursed for one month of summer expenses (8\% effort) starting in year 1.  Accounting for a merit increase in year 2, this rate amounts to \$7,869 per month in year 1, \$8,442 per month in year 2 and \$8,696 per month in year 3.  Co-PI Richard Grotjahn will be reimbursed for two weeks of summer expenses (4\% effort) starting in year 1.  This rate amounts to \$6,861 per month in year 1, \$7,067 per month in year 2 and \$7,279 per month in year 3.

\textit{Graduate Student Researchers} \\
Two graduate student researchers (GSR4; 9 months at 48\%, 3 summer months at 100\%) will be funded for the duration of this project, conducted as part of their degree with a thesis option.  The annual salary rate for a GSR IV is \$45,300 for the first year, with inflation applied in subsequent years.  This amounts to \$28,695 in year 1, \$29,556 in year 2 and \$30,442 in year 3.

\section{Benefits}
\vspace{-0.3cm}

Benefit rates for the project are standard UC Davis rates, as follows:  Faculty summer salary (17.0\% in year 1, 18.0\% in year 2, 18.5\% in year 3) and Graduate Student Researchers (1.3\%).

\section{Travel}
\vspace{-0.3cm}

The budget includes domestic travel costs associated with three annual trips to the American Geophysical Union (AGU) Fall Meeting in San Francisco, CA for the PI and two graduate students.  Including registration fees, travel expenses and per-diem costs, the total cost is estimated at \$1400 per traveler per year.  Domestic travel expenses are also included for the PI to present at the American Meteorological Society (AMS) 30th annual Climate Variability and Change conference, to be held in January 2018, estimated at \$2300 in year 3.  International travel expenses are included to the International Association of Meteorology and Atmospheric Sciences (IAMAS) conference in Cape Town, South Africa, to be held in 2017, estimated at \$4400 for one participant.

\section{Supplies}
\vspace{-0.3cm}

A one-time expense of \$3550 will cover the cost of 50 TB in external hard drives to store reanalysis and model data that is generated or utilized by this project.

\section{Sub-awards}
\vspace{-0.3cm}

This proposal will award \$4200 per year to Colin Zarzycki (National Center for Atmospheric Research), covering annual travel expenses to the American Geophysical Union (AGU) Fall Meeting (\$1400 per year) and one trip to UC Davis for consultation / in-person meetings (\$1250 per year).  A National Center for Atmospheric Research (NCAR) overhead rate of XX\% is included in this calculation.

\section{Other Direct Costs}
\vspace{-0.3cm}

\subsection{Publication Charges}

Publication costs are incurred from publication of work produced by this project.  The estimated cost is \$2,000 per year (16 pages at \$125 per page).

\subsection{Software Licenses}

Software license charges are \$138 / year / license for the mathematics software package Maple and \$165 / year / license for the software package Matlab, for a total of \$303 / year / license.  Software will be provided to the GSRs (2 licenses) and PI (1 license) over the duration of the project.  This amounts to \$909 / year for all licenses.  These software packages will be used by the students and PI for data processing, analysis and modeling.

\subsection{Tuition}

Resident tuition expenses for the GSR at UC Davis are \$16,541 in year 1 as a consequence of a discounted tuition rate from the Provost, escalated at a rate of 10\% per year.  For both GSRs, this amounts to \$27,292 in year 1, \$30,021 in year 2 and \$33,023 in year 3.

\section{Indirect Cost}
\vspace{-0.3cm}

Per the University of California, Davis� Federally approved Indirect Cost Rate for on-campus research the rate of 56.5\% Modified Total Direct Cost will be applied for the period of September 1, 2015 through June 30, 2016 and 57.0\% for July 1, 2016 through August 31, 2018.

\end{document}
